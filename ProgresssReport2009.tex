\documentclass[a4paper,twoside,english,DIV=calc]{scrartcl}

\usepackage{graphicx}
\usepackage[utf8]{inputenc}
\usepackage[T1]{fontenc}
\usepackage{babel}
\usepackage{siunitx}
\usepackage{svn-multi}
\usepackage{subfig}
\usepackage{fancyhdr}
\usepackage{microtype}
\usepackage{scrtime}
\usepackage{tikz}
\usepackage{pgfplots}
\usepackage[numbers,square,sort&compress]{natbib}
\usepackage{booktabs}
\usepackage{todonotes}
\usepackage{lineno}
	\linenumbers
	\modulolinenumbers[2]

\usepackage[backref]{hyperref}
	\hypersetup{%
		pdfauthor={David Haberthür},%
		pdftitle={Progress Report for the Graduate School in Cellular and Biomedical Sciences - 2009},%
		pdfsubject={Progress Report},%
		pdfkeywords={progress report, graduate school, report},%
		pdfpagemode=UseThumbs,
		colorlinks=true % ... colored (true > no border, use "false" for final output)
		}
\usepackage[all]{hypcap}% make hyperref work nicely with captions

%% Subversion Information
\svnidlong
{$HeadURL$}
{$LastChangedDate$}
{$LastChangedRevision$}
{$LastChangedBy$}
\svnid{$Id$} 

\pagestyle{fancy}
\fancyfoot{}
\fancyfoot[OR]{\tiny \href{\svnkw{HeadURL}}{Revision \svnkw{LastChangedRevision}} last saved on \svnkw{LastChangedDate} --- page \thepage}
\fancyfoot[EL]{\tiny page \thepage --- \href{\svnkw{HeadURL}}{Revision \svnkw{LastChangedRevision}} last saved on \svnkw{LastChangedDate}}

\newcommand{\imsize}{\linewidth}

\newlength\imagewidth
\newlength\imagescale

\title{Progress Report for the Graduate School in Cellular and Biomedical Sciences - 2009}
\subtitle{(SVN-revision: \svnkw{LastChangedRevision}, compiled \today, \thistime)}
\author{David Haberthür}
\date{\today}

\begin{document}
\maketitle

\section{Introduction}
Again, this year the focus of my work is on on three-dimensional datasets obtained with high resolution synchrotron radiation based x-ray tomographic microscopy (SRXTM) at the beamline for TOmographic Microscopy and Coherent rAdiology experimenTs (TOMCAT) at the Swiss Light Source (SLS), Paul Scherrer Institute (PSI) in Villigen, Switzerland. During several beamtimes (details see section~\ref{sec:srxtm}) we recorded datasets of rat and mice lung samples with different characteristics. In the past year I have presented my work in multiple talks at both national and international conferences and have presented several posters about the topics I have been working on.

\section{SRXTM}\label{sec:srxtm}
During 6 regularly allotted beam times we obtained tomographc datasets of 170 samples, an increase in output compared to the output presented in the last progress report. In total we performed nearly 248 single scans; the difference between amount of tomographic datasets and single scans again arises through the use of the so called wide field scanning (see section~\ref{sec:wfs}), where multiple single scans are merged to obtain tomographic datasets with increased field of view at the same pixel-resolution as the single scans.

Apart from 4 scans where we calibrated the tomographic beamline with electron microscopy grids or needle pins to thoroughly assess the beam and beamline characteristics, all tomographic datasets scanned have been of rat and mice lung samples. Slightly less than half of the tomographic scans (%
% 15 Mariani
% 12 Gold 
% 36 Stockx
% 13 Sophie
76) have been scanned as part of four ongoing collaborations with external partners. The collaborations are:
\begin{description}
\item[The Lung Biology Center, Harvard Medical School] In the ongoing collaboration with Thomas Mariani we scanned 15 samples of PPAR Knockout mice. The tomographic datasets are used to study the alterations of the airway diameter in the mice. A paper of the findings is in preparation.
\item[GSF-Institute of Inhalation Biology, Munich] Twelve samples of rat lungs with inhaled gold particles of \SI{200}{\nano\meter} and \SI{700}{\nano\meter} in size have been scanned. The tomographic datasets are used for a study on a bronchiopulmonary disease model as a preparation of an application of the  NFP 64 «Chancen und Risiken von Nanomaterialien». Generally, we use absorption-contrast tomography, but for these samples we also started to use phase-contrast tomography, which enables us to use unstained lung samples, which permits us to perform immunostaining and other biologically active test after obtaining tomographic datasets, something which is not possible with the uranyl-acetate-stained luns samples used for absorption-contrast tomography.
\item[Medicine, Nursing and Health Sciences, Monash University] For a study to investigate the effects of different activated Protein C isoforms and hyperoxia exposure on the pulmonary vasculature and capillary density of rat lungs we scanned 36 samples. 
\item[Lehrstuhl für Numerische Mechanik, TU Munich] As a provider of the data for a study on the mechanics of alveolar tissue under breathing and motion stress we scanned 13 samples.
\end{description}

\begin{table}[htp]
\centering
\scriptsize
\begin{tabular}{clc}
\toprule
Beamtime & Details & Total\\
\midrule
09a & 15 Mariani, 1 Pins, 5 Normal, 4 WFS (12) & 25 (33)\\
09b & 3 Grid, 7 MBA, 12 Normal, 4 WFS (12) & 26 (34)\\
09c & 2 Normal, 12 WFS (36) & 14 (38)\\
09d & 1 Normal, 3 Stacked (17), 6 WFS (18), 2 Stacked MBA (6) & 30 (42)\\
09e & Gold (4 Normal, 4 Edge, 4 MBA), 2 WFS (6), Stockx 18 Stacked (36) & 50 (54)\\
09f & 13 Stacked Sophie, 11 WFS (33), 1 Normal & 25 (47)\\
\cmidrule(lr){3-3}
& & 170 (248)\\
\bottomrule
\end{tabular}
\normalsize
\caption{Details of all Beamtimes}
\label{tab:srxtm}
\end{table}

\section{Wide field scanning}
The method for increasing the field of view of the TOMCAT beamline presented in the last progress report has been routinely applied for the imaging of rat lung samples by our group and is still pending the full implementation as a module into the beamline workflow at the SLS. A lot of my working time has been spent on the preparation of a manuscript for submission to the Journal of Synchrotron Radiation, which explains the concepts and implementation of the socalled wide field x-ray tomographic microscopy (WF-SRXTM). This manuscript is to be submitted by now\todo{really?}. The technical side of the wide field scanning method has been covered, future work will focus on the answering of biological questions using this method. Using WF-SRXTM we can now visualize entire rat lung acini\footnote{The acinus is defined as the complex of alveolated airways distal to the terminal bronchiole \cite{Rodriguez1987}, in which the gas-exchange in the lung takes place.} , something which was not generally possible using classical SXTM, as can be seen in figure~\ref{fig:srxtm_vs_wfsrxtm}.

\renewcommand{\imsize}{0.5\linewidth}
\begin{figure}[htp]
	\subfloat[Conventional SRXTM]{%
		\pgfmathsetlength{\imagewidth}{\imsize}%
		\pgfmathsetlength{\imagescale}{\imagewidth/1202}%
		\begin{tikzpicture}[x=\imagescale,y=-\imagescale]
			\def\x{293} % scalebar-x at golden ratio of x=1202px
			\def\y{612} % scalebar-y at 90% of height of y=680px
			\node[anchor=north west,inner sep=0pt,outer sep=0pt] at (0,0)
				{\includegraphics[width=\imagewidth]{img/R108C04C-overview-s2}};
			\draw[|-|,thick] (\x,\y) -- (\x+116,\y) node [midway,above] {\SI{500}{\micro\meter}};
		\end{tikzpicture}%
		\label{subfig:srxtm}%
		}%
	\subfloat[Wide Field SRXTM with enhanced field of view]{%
		\pgfmathsetlength{\imagewidth}{\imsize}%
		\pgfmathsetlength{\imagescale}{\imagewidth/1202}%
		\begin{tikzpicture}[x=\imagescale,y=-\imagescale]
			\def\x{243} % scalebar-x at golden ratio of x=1202px
			\def\y{612} % scalebar-y at 90% of height of y=680px
			\node[anchor=north west,inner sep=0pt,outer sep=0pt] at (0,0)
				{\includegraphics[width=\imagewidth]{img/R108C04C-overview-merge}};
			\draw[|-|,thick] (\x,\y) -- (\x+232,\y) node [midway,above] {\SI{1}{\milli\meter}};
		\end{tikzpicture}%
		\label{subfig:wf-srxtm}%
		}%
	\label{fig:srxtm_vs_wfsrxtm}%
	\caption{Comparison of conventiuonal SRXTM with WF-SRXTM of the same rat lung sample, obtained at postnatal day 4. Independent airway segments have been extracted using a region growing algorithm. With the conventional SRXTM (fig.~\ref{subfig:srxtm}) we see that the red airway segment is not fully contained in the total field of view of approx.\,1.52$\times$1.52$\times$\SI{1.52}{\milli\meter}. The green airway segment is also only partially contained. Increasing the field of view to approx.\,3.7$\times$3.7$\times$\SI{1.52}{\milli\meter}  %881x2516x1024px
(fig~\ref{subfig:wf-srxtm}) adds more distinguishable airway segments to the tomographic dataset. We now see that the green airway segment is fully contained in the volume and an additional, yellow marked volume becomes visible.
}%
\end{figure}

One preliminary result found with the visualization of rat lung acini over the postnatal lung development (day 4, 10, 21, 36 and 60 after birth) is that the acinus grows more than expected and predicted up to now. Even if we Albeit we have increased the visible sample volume nine-fold, we are still not able to fully visualize an entire rat lung acinus for the later development days. We thus recently recorded so called stacked WF-SRXTM scans, which increase the field of view both horizontally and vertically and plan to use these datasets for the study of the postnatal development of the acinar size.

The integration of the developed workflow into a program accessible by all users of the TOMCAT beamline is still pending, but should be happening in the first half year of 2010. 

\section{Multimodal Imaging}
The manuscript attached to the last progress report has been published in the Journal of Physics~\cite{Haberthuer2009}. The initial workflow for multimodal imaging---the combination of the full, unconstricted three-dimensional information of SRXTM datasets with the ultrahigh resolution of transmission electron microscopy (TEM) images---which has been presented in the aforementioned manuscript has been perfected by a master student in our group during his six-month master thesis. We are now able to routinely register datasets from two differentimaging modalities. Two-dimensional ultrahigh resolution TEM slices without information on the exact location in the airway tree can now be matched with three-dimensional high-resolution SRXTM slices containing the unrestricted information on the exact location in the airway tree. This enables us to exaxtly study and visualize particle deposition locations both on a cellular level and on the level of the location in the airway tree. Partial support on this master thesis has been made by me.

\section{Structural Information}
\begin{itemize}
\item Skeletonization is still problematic. Not in terms of Algorithm, but in terms of biological correctness. Wahlfachpraktikums-Studenten worked on this. Study with "real" stereological counting is planned (Wahlfach2010)
\item Meeting with Fraunhofer MeVis in October for adaptation of algorithm from tubular to other structures. Scanco-Algorithm at PSI can NOT do anything > Work of Xris for paper
\item extraction of information, we've seen that the acinus gets bigger over development > idea for next publication
\end{itemize}

The extraction of the airway structure of the terminal airways has been refined. - Last year presented small acinus - now much bigger - other problems (not whole acinus in dataset $\rightarrow$ stacked wide field scan) - work/meeting with MeVis

For further analysis four regions of interest with a side length of 256 pixels (at \SI{1.48}{\micro\meter\per pixel}, thus containing a volume of \SI{1.678e7}{voxels}) have been extracted for each of the protocols B, L and T. The three-dimensional placement of these ROIs inside the sample is shown in figure~\ref{fig:roi3d}. Each of the ROIs has been binarized using an algorithmically determined threshold~\cite{Otsu1979} and small particles inside the segmented airspace lumen have been removed using a connected components analysis. Subsequently, the euclidean distance transformation has been calculated for each thresholded ROI.

\renewcommand{\imsize}{\columnwidth}
\begin{figure}[htp]
	\centering
	\pgfmathsetlength{\imagewidth}{\imsize}
	\pgfmathsetlength{\imagescale}{\imagewidth/1452}
	\begin{tikzpicture}[x=\imagescale,y=-\imagescale]
		\def\x{297} % scalebar-x at golden ratio of x=1452px
		\def\y{684} % scalebar-y at 90% of height of y=760px
		\node[anchor=north west, inner sep=0pt, outer sep=0pt] at (0,0)
	     {\includegraphics[width=\imagewidth]{img/ROIs-3d}};
		% 1357px = 4.0138mm > 100px = 296um > 169px = 500um
		% \draw[|-|,thick,red] (83,517) -- (1425,719) node [sloped,midway,below] {\SI{4.0138}{\milli\meter} (2712px)};
		\draw[|-|,thick] (\x,\y) -- (\x+169,\y) node [midway, above] {\SI{500}{\micro\meter}};
		%\draw[|-|,thick, white] (\x+645,\y-180) -- (\x+645+128,\y-180) node [midway, below] {\SI{256}{pixels}};
		\draw ( 368,360) node [fill=white, semitransparent] {ROI 1} node {ROI 1};
		\draw (1038,312) node [fill=white, semitransparent] {ROI 2} node {ROI 2};
		\draw ( 767,413) node [fill=white, semitransparent] {ROI 3} node {ROI 3};
		\draw ( 684,139) node [fill=white, semitransparent] {ROI 4} node {ROI 4};
		\end{tikzpicture}%
	\caption{Overview of the placement of the four regions of interest where the histogram of the euclidean distance transformation distribution has been calculated. Grey: Semitransparent volume rendering of the lung tissue sample. Red: Four regions of interest, extracted to calculate the distance transformation, each with a side-length of 256 pixels. The labels of the ROIs conform to the legends in figure~\ref{fig:DTFplots}.}%		
	\label{fig:roi3d}
\end{figure}

For comparison, the histogram of the euclidean distance transformation has been plotted for all four regions of interest in each protocol (B, L and T).

\renewcommand{\imsize}{.309\columnwidth}
\begin{figure}[htp]
	\centering
	\begin{tabular}{cc}
		\input{img/edited-plot-roi1}&%
		\input{img/edited-plot-roi2}\\%
		\input{img/edited-plot-roi3}&%
		\input{img/edited-plot-roi4}%
	\end{tabular}
	\caption{Histogram-Plots for each of the of 4 ROIs, each showing the histogram of the distance transformation for the protocols B, L and T.}%
	\label{fig:DTFplots}
\end{figure}

\subsection{Skeletonization}
Extraction of Number of Nodes and Number of Edges - matching with volume-fraction to gain information about structural alteration during lung development

\section{Publications}\label{sec:publications}
\subsection{Publication on the enhancement of the field of view of TOMCAT}
The planned publication on the enhancement of the field of view of TOMAT has been delayed because of several problems. After a first version of the manuscript has been delayed because of multiple conferences and been set aside, a second version has been prepared for discussion with the co-authors. During this discussion we agreed on a slightly changed direction of the paper, which required waiting for new data from one of the co-authors. After two months we concluded that it's not possible to generate the new data the way we hoped. This data has now been generated differently and the manuscript is in the final editing stages ans should be submitted shortly.

\subsection{Publication on bone scans performed last year}
Last year we made several tomographic scans of mice bones (tibia, femur and vertebrae) as part of a collaborative study. The manuscript which resulted from this collaboration has been submitted to Nature Medicine, details can be found in citation~\cite{Sausbier2009b}.

\bibliographystyle{unsrtnat}
\bibliography{../../references}

\end{document}